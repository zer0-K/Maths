% ===== Packages =====
\usepackage[margin=1in]{geometry}
\usepackage{amsmath, amssymb, mathtools}
\usepackage{tikz-cd}  % For commutative diagrams
\usepackage{stmaryrd} % For \llbracket, \rrbracket

% --- general features ---
\setlength{\parindent}{1em} % Standard indent (~2 chars)
\setlength{\parskip}{0pt}   % No extra spacing

% ===== Custom Commands =====
% --- Syntax ---
\newcommand{\var}[1]{{#1}}         % variable
\newcommand{\name}[1]{{#1}}        % name

% --- General Math ---
\newcommand{\zero}{0}              % Empty type
\newcommand{\unit}{1}              % Unit type
\newcommand{\N}{\mathbb{N}}        % Natural numbers
\newcommand{\cat}[1]{\mathbf{#1}}  % Categories (bold)

% --- Category Theory ---
\newcommand{\op}[1]{#1^{\mathrm{op}}}          % Opposite category
\newcommand{\functor}[1]{\mathscr{#1}}         % Functor (script)
\newcommand{\Hom}[3]{\mathrm{Hom}_{#1}(#2,#3)} % Hom-set

% --- Type Theory ---
\newcommand{\U}[1]{\mathcal{U}_{#1}}           % universe
\newcommand{\context}[1]{{#1}}                 % context
\newcommand{\ctx}{\Gamma}                      % Gamma context
\newcommand{\Id}[3]{{#2}=_{#1}{#3}}            % Identity type
\newcommand{\refl}[1]{\text{refl}_{#1}}        % universe

% --- HoTT/Homotopy Theory ---
\newcommand{\isContr}{\mathsf{isContr}}        % Is contractible
\newcommand{\eqv}[2]{#1 \simeq #2}             % Equivalence

% --- Code/Programming ---
\newcommand{\code}[1]{\texttt{#1}}             % Monospace font

% --- For latex parsing ---
\newcommand{\probe}{}
% definition
\newcommand{\notation}[1]{#1}
\newcommand{\actualDef}[1]{#1}
\newenvironment{definition}[2][Definition]{\begin{trivlist}
\item[\hskip \labelsep \textbf{#1}\hskip \labelsep \textbf{#2.}]}{\end{trivlist}}
% axiom
\newcommand{\axiomName}[1]{#1}
\newcommand{\actualAxiom}[1]{#1}
\newenvironment{axiom}[2][Axiom]{\begin{trivlist}
\item[\hskip \labelsep \textbf{#1}\hskip \labelsep \textbf{#2.}]}{\end{trivlist}}
\newcommand{\axiomApplyEmpty}[1]{We apply axiom {#1}}
\newcommand{\axiomApply}[2]{We apply axiom {#1} with {#2}}
% context environment
\newcommand{\ctxenvName}[1]{#1}
\newcommand{\actualCtxenv}[1]{#1}
\newenvironment{context_env}[2][Context environment]{\begin{trivlist}
\item[\hskip \labelsep \textbf{#1}\hskip \labelsep \textbf{#2.}]}{\end{trivlist}}
% construction
\newcommand{\constructionName}[1]{#1}
\newcommand{\actualConstruction}[1]{#1}
\newenvironment{construction}[2][Construction]{\begin{trivlist}
\item[\hskip \labelsep \textbf{#1}\hskip \labelsep \textbf{#2.}]}{\end{trivlist}}
% usual maths commands
\newenvironment{theorem}[2][Theorem]{\begin{trivlist}
\item[\hskip \labelsep \textbf{#1}\hskip \labelsep \textbf{#2.}]}{\end{trivlist}}
\newenvironment{lemma}[2][Lemma]{\begin{trivlist}
\item[\hskip \labelsep \textbf{#1}\hskip \labelsep \textbf{#2.}]}{\end{trivlist}}
\newenvironment{exercise}[2][Exercise]{\begin{trivlist}
\item[\hskip \labelsep \textbf{#1}\hskip \labelsep \textbf{#2.}]}{\end{trivlist}}
\newenvironment{problem}[2][Problem]{\begin{trivlist}
\item[\hskip \labelsep \textbf{#1}\hskip \labelsep \textbf{#2.}]}{\end{trivlist}}
\newenvironment{question}[2][Question]{\begin{trivlist}
\item[\hskip \labelsep \textbf{#1}\hskip \labelsep \textbf{#2.}]}{\end{trivlist}}
\newenvironment{corollary}[2][Corollary]{\begin{trivlist}
\item[\hskip \labelsep \textbf{#1}\hskip \labelsep \textbf{#2.}]}{\end{trivlist}}

\newenvironment{solution}{\begin{proof}[Solution]}{\end{proof}}
 