\documentclass[12pt]{article}
 
\usepackage{subfiles}
% ===== Packages =====
\usepackage[margin=1in]{geometry}
\usepackage{amsmath, amssymb, mathtools}
\usepackage{tikz-cd}  % For commutative diagrams
\usepackage{stmaryrd} % For \llbracket, \rrbracket

% --- general features ---
\setlength{\parindent}{1em} % Standard indent (~2 chars)
\setlength{\parskip}{0pt}   % No extra spacing

% ===== Custom Commands =====
% --- Syntax ---
\newcommand{\var}[1]{{#1}}         % variable

% --- Inference system ---
\newcommand{\context}[1]{{#1}}     % context

% --- General Math ---
\newcommand{\N}{\mathbb{N}}        % Natural numbers
\newcommand{\cat}[1]{\mathbf{#1}}  % Categories (bold)

% --- Category Theory ---
\newcommand{\op}[1]{#1^{\mathrm{op}}}          % Opposite category
\newcommand{\functor}[1]{\mathscr{#1}}         % Functor (script)
\newcommand{\Hom}[3]{\mathrm{Hom}_{#1}(#2,#3)} % Hom-set

% --- Type Theory ---
\newcommand{\U}[1]{\mathcal{U}_{#1}}           % universe
\newcommand{\deduce}{\vdash}                   % Judgement line
\newcommand{\ctx}{\Gamma}                      % Context

% --- HoTT/Homotopy Theory ---
\newcommand{\isContr}{\mathsf{isContr}}        % Is contractible
\newcommand{\eqv}[2]{#1 \simeq #2}             % Equivalence

% --- Code/Programming ---
\newcommand{\code}[1]{\texttt{#1}}             % Monospace font

% --- For latex parsing ---
\newcommand{\probe}{}
% definition
\newcommand{\notation}[1]{#1}
\newcommand{\actualDef}[1]{#1}
\newenvironment{definition}[2][Definition]{\begin{trivlist}
\item[\hskip \labelsep \textbf{#1}\hskip \labelsep \textbf{#2.}]}{\end{trivlist}}
% axiom
\newcommand{\axiomName}[1]{#1}
\newcommand{\actualAxiom}[1]{#1}
\newenvironment{axiom}[2][Axiom]{\begin{trivlist}
\item[\hskip \labelsep \textbf{#1}\hskip \labelsep \textbf{#2.}]}{\end{trivlist}}
%
\newenvironment{theorem}[2][Theorem]{\begin{trivlist}
\item[\hskip \labelsep \textbf{#1}\hskip \labelsep \textbf{#2.}]}{\end{trivlist}}
\newenvironment{lemma}[2][Lemma]{\begin{trivlist}
\item[\hskip \labelsep \textbf{#1}\hskip \labelsep \textbf{#2.}]}{\end{trivlist}}
\newenvironment{exercise}[2][Exercise]{\begin{trivlist}
\item[\hskip \labelsep \textbf{#1}\hskip \labelsep \textbf{#2.}]}{\end{trivlist}}
\newenvironment{problem}[2][Problem]{\begin{trivlist}
\item[\hskip \labelsep \textbf{#1}\hskip \labelsep \textbf{#2.}]}{\end{trivlist}}
\newenvironment{question}[2][Question]{\begin{trivlist}
\item[\hskip \labelsep \textbf{#1}\hskip \labelsep \textbf{#2.}]}{\end{trivlist}}
\newenvironment{corollary}[2][Corollary]{\begin{trivlist}
\item[\hskip \labelsep \textbf{#1}\hskip \labelsep \textbf{#2.}]}{\end{trivlist}}

\newenvironment{solution}{\begin{proof}[Solution]}{\end{proof}}
  


\begin{document}

\title{Foundations - type theory}
 
\maketitle
\tableofcontents

% ------------------------
% --- Deductive system ---
% ------------------------
\section{Deductive system}



% contexts
\subsection{Contexts}

Here are the basic principles for contexts in type theory.

\begin{definition}{0.1.1}
We will note \notation{$.$} \probe for the \actualDef{empty context} \probe
\end{definition}

\begin{axiom}{0.1.1}
\axiomName{empty derivation} \probe :

\actualAxiom{$rule\_empty\_derivation : \epsilon\ \text{——}\ empty\_context\ ctx$} \probe
\end{axiom}

\begin{axiom}{0.1.2}
\axiomName{context extension} \probe : 

\actualAxiom{$rule\_context\_extension : \context{\Gamma} \vdash \var{A}:\U{i},\quad \var{x}\ not\ in\ dom(\context{\Gamma})\ \text{——}\ \context{\Gamma},\ \var{x}:\var{A}\ ctx$} \probe
\end{axiom}

\begin{axiom}{0.1.3}

\axiomName{variable rule} \probe : 
\actualAxiom{$rule\_variable : \context{\Gamma}\ ctx,\quad \var{x}:\var{A}\ in\ \context{\Gamma}\ \text{——}\ \context{\Gamma} \vdash \var{x}:\var{A}$} \probe
\end{axiom}



% judgmental equalities
\subsection{Judgemental equalities}

We suppose that judgmental equality is an equivalence relation. Namely:

\begin{axiom}{0.2.1}
\axiomName{Judgmental reflexivity} \probe : 

\actualAxiom{$rule\_judg\_refl : \context{\Gamma} \vdash \var{a}:\var{A}\ \text{——}\ \context{\Gamma} \vdash \var{a} \equiv \var{a}:\var{A}$} \probe
\end{axiom}

\begin{axiom}{0.2.2}
\axiomName{Judgmental symmetry} \probe : 

\actualAxiom{$rule\_judg\_sym : \context{\Gamma} \vdash \var{a}\equiv \var{b}:\var{A}\ \text{——}\ \context{\Gamma} \vdash \var{b} \equiv \var{a}:\var{A}$} \probe
\end{axiom}

\begin{axiom}{0.2.3}
\axiomName{Judgmental transitivity} \probe : 

\actualAxiom{$rule\_judg\_trans : \context{\Gamma} \vdash \var{a}\equiv \var{b}:\var{A},\quad \context{\Gamma} \vdash \var{b}\equiv \var{c}:\var{A}\ \text{——}\ \context{\Gamma} \vdash \var{a} \equiv \var{c}:\var{A}$} \probe
\end{axiom}

\begin{axiom}{0.2.4}
\axiomName{Judgmental typing} \probe : 

\actualAxiom{$rule\_judg\_typing : \context{\Gamma} \vdash \var{a}:\var{A},\quad \context{\Gamma} \vdash \var{A}\equiv \var{B}:\U{i}\ \text{——}\ \context{\Gamma} \vdash \var{a}:\var{B}$} \probe
\end{axiom}

\begin{axiom}{0.2.5}
\axiomName{Judgmental typing equiv} \probe : 

\actualAxiom{$rule\_judg\_typing\_eq : \context{\Gamma} \vdash \var{a}\equiv \var{b}:\var{A},\quad \context{\Gamma} \vdash \var{A}\equiv \var{B}:\U{i}\ \text{——}\ \context{\Gamma} \vdash \var{a}\equiv \var{b}:\var{B}$} \probe
\end{axiom}



% Universes
\subsection{Type universe}

\begin{axiom}{0.3.1}
\axiomName{Universe intro} \probe : 

\actualAxiom{$rule\_univ\_intro : \context{\Gamma}\ ctx\ \text{——}\ \context{\Gamma} \vdash \U{i}:\U{i+1}$} \probe
\end{axiom}

\begin{axiom}{0.3.2}
\axiomName{Universe cumul} \probe : 

\actualAxiom{$rule\_univ\_cumul : \context{\Gamma} \vdash \var{A}:\U{i}\ \text{——}\ \context{\Gamma} \vdash \var{A}:\U{i+1}$} \probe
\end{axiom}





% --------------------
% --- Type formers ---
% --------------------
\section{Type formers}



% Function type
\subsection{Function type}

\begin{axiom}{1.1.1}
\axiomName{Function formation} \probe :

\actualAxiom{$rule\_\lambda-form : \context{\Gamma} \vdash \var{A}:\U{i},\quad \context{\Gamma}, \var{a}:\var{A} \vdash \var{B}:\U{i}\ \text{——}\ \context{\Gamma} \vdash \var{A} \to \var{B}:\U{i}$} \probe
\end{axiom}

\begin{axiom}{1.1.2}
\axiomName{Function introduction} \probe : 

\actualAxiom{$rule\_\lambda-intro : \context{\Gamma}, \var{a}:\var{A} \vdash \var{B}:\U{i}\ \text{——}\ \context{\Gamma} \vdash \lambda (\var{x}:\var{A}).\var{b}:\var{A} \to \var{B}$} \probe
\end{axiom}

\begin{axiom}{1.1.3}
\axiomName{Function elimination} \probe : 

\actualAxiom{$rule\_\lambda-elim : \context{\Gamma} \vdash \var{f}:\var{A}\to\var{B},\quad \context{\Gamma} \vdash \var{a}:\var{A}\ \text{——}\ \context{\Gamma} \vdash \var{f(a)}:\var{B}$} \probe
\end{axiom}

\begin{axiom}{1.1.4}
\axiomName{Function computation} \probe : 

\actualAxiom{$rule\_\lambda-comp : \context{\Gamma}, \var{x}:\var{A} \vdash \var{b}:\var{B},\quad \context{\Gamma} \vdash \var{a}:\var{A}\ \text{——}\ \context{\Gamma} \vdash \text{let}\ \var{x} = \var{a}\ \text{in}\ \lambda (\var{x}:\var{A}).\var{b}:\var{B}$} \probe
\end{axiom}

\begin{axiom}{1.1.5}
\axiomName{Function uniqueness principle} \probe : 

\actualAxiom{$rule\_\lambda-uniq : \context{\Gamma} \vdash \var{f}:\var{A}\to\var{B}\ \text{——}\ \context{\Gamma} \vdash \var{f} \equiv \lambda (\var{x}:\var{A}).\var{f}(\var{x}):\var{A} \to \var{B}$} \probe
\end{axiom}



% Dependent function type
\subsection{Dependent function type}

\indent
Similar as non dependent functions.

\begin{axiom}{1.2.1}
\axiomName{Dependent function formation} \probe : 

\actualAxiom{$rule\_\Pi-form : \context{\Gamma} \vdash \var{A}:\U{i},\quad \context{\Gamma}, \var{a}:\var{A} \vdash \var{B}:\U{i}\ \text{——}\ \context{\Gamma} \vdash \Pi_{\var{x}:\var{A}}\var{B}(\var{x}):\U{i}$} \probe
\end{axiom}

\begin{axiom}{1.2.2}
\axiomName{Dependent function introduction} \probe : 

\actualAxiom{$rule\_\Pi-intro : \context{\Gamma}, \var{a}:\var{A} \vdash \var{B}:\U{i}\ \text{——}\ \context{\Gamma} \vdash \lambda (\var{x}:\var{A}).\var{b}:\Pi_{\var{x}:\var{A}}\var{B}(\var{x})$} \probe
\end{axiom}

\begin{axiom}{1.2.3}
\axiomName{Dependent function elimination} \probe : 

\actualAxiom{$rule\_\Pi-elim : \context{\Gamma} \vdash \var{f}:\Pi_{\var{x}:\var{A}}\var{B},\quad \context{\Gamma} \vdash \var{a}:\var{A}\ \text{——}\ \context{\Gamma} \vdash \var{f(a)}:\var{B}(\var{a})$} \probe
\end{axiom}

\begin{axiom}{1.2.4}
\axiomName{Dependent function computation} \probe : 

\actualAxiom{$rule\_\Pi-comp : \context{\Gamma}, \var{x}:\var{A} \vdash \var{b}:\var{B},\quad \context{\Gamma} \vdash \var{a}:\var{A}\ \text{——}\ \context{\Gamma} \vdash \text{let}\ \var{x} = \var{a}\ \text{in}\ \lambda (\var{x}:\var{A}).\var{b}:\var{B}(\var{x})$} \probe
\end{axiom}

\begin{axiom}{1.2.5}
\axiomName{Dependent function uniqueness principle} \probe : 

\actualAxiom{$rule\_\Pi-uniq : \context{\Gamma} \vdash \var{f}:\Pi_{\var{x}:\var{A}}\var{B}\ \text{——}\ \context{\Gamma} \vdash \var{f} \equiv \lambda (\var{x}:\var{A}).\var{f}(\var{x}):\Pi_{\var{x}:\var{A}}\var{B}$} \probe
\end{axiom}



\end{document}